\documentclass{article}
\usepackage[utf8]{inputenc}
\usepackage[T1]{fontenc}
\usepackage[french]{babel}
\usepackage{graphicx}
\usepackage{xcolor}
\usepackage{hyperref}

\title{Description du Jeu: KeyScale}
\author{Documentation LaTeX}
\date{\today}

\begin{document}

\maketitle

\section{Introduction}
\textit{KeyScale} est un jeu d'arcade développé en Python avec Pygame où le joueur doit taper des lettres qui apparaissent à l'écran pour grimper et échapper à la lave montante. Le jeu combine les mécaniques de dactylographie et de réflexes pour créer une expérience à la fois éducative et divertissante.

\section{Mécaniques de Jeu}

\subsection{Concept Principal}
Le joueur contrôle un personnage qui doit grimper pour éviter la lave qui monte progressivement à l'écran. Pour grimper, le joueur doit taper les lettres qui apparaissent dans des carrés. Chaque lettre correctement tapée fait monter le personnage et éloigne temporairement la lave.

\subsection{Éléments de Jeu}
\begin{itemize}
    \item \textbf{Lettres à taper:} Des carrés contenant des lettres aléatoires apparaissent et descendent à l'écran.
    \item \textbf{Pièges:} Certains carrés sont des pièges qui, lorsqu'activés, font tomber des rochers et coûtent une vie au joueur.
    \item \textbf{Lave:} Un élément qui monte progressivement et que le joueur doit éviter.
    \item \textbf{Vies:} Le joueur dispose d'un nombre limité de vies (4 par défaut).
    \item \textbf{Score:} Augmente à chaque lettre correctement tapée.
\end{itemize}

\subsection{Niveaux de Difficulté}
Le jeu propose trois niveaux de difficulté:
\begin{itemize}
    \item \textbf{Facile:} Lave plus lente, plus de délai avant son apparition, lettres plus espacées.
    \item \textbf{Moyen:} Vitesse modérée, délai moyen avant l'apparition de la lave.
    \item \textbf{Difficile:} Lave rapide, peu de délai, lettres plus fréquentes.
\end{itemize}

\section{Version Console (CLI)}
En plus de la version graphique, \textit{KeyScale} propose une version en ligne de commande (CLI) qui conserve l'essence du jeu tout en utilisant une interface textuelle.

\subsection{Principe du Mode CLI}
Dans cette version minimaliste, le joueur est confronté à une expérience de dactylographie pure:
\begin{itemize}
    \item Des lettres apparaissent aléatoirement dans la console
    \item Le joueur doit taper ces lettres le plus rapidement possible
    \item Le temps de réaction est mesuré pour chaque lettre
    \item Un score est calculé en fonction de la vitesse et de la précision
\end{itemize}

\subsection{Fonctionnalités Spécifiques à la Version CLI}
\begin{itemize}
    \item \textbf{Visibilité du temps:} Affichage du temps écoulé pour chaque saisie
    \item \textbf{Mode entrainement:} Possibilité de s'exercer sur des combinaisons de lettres spécifiques
    \item \textbf{Statistiques détaillées:} Analyse des performances par touche du clavier
    \item \textbf{Accessibilité:} Compatible avec les lecteurs d'écran pour les malvoyants
\end{itemize}

\subsection{Avantages du Mode CLI}
Cette version alternative offre plusieurs avantages:
\begin{itemize}
    \item Utilisation possible sur des systèmes aux ressources limitées
    \item Concentration maximale sur les compétences de frappe sans distraction visuelle
    \item Excellente option pour l'entraînement dactylographique pur
    \item Possibilité de l'utiliser via SSH ou sur des terminaux distants
\end{itemize}

\section{Interface Utilisateur}

\subsection{Écrans Principaux}
\begin{itemize}
    \item \textbf{Menu Principal:} Permet de commencer une partie, changer les paramètres ou quitter.
    \item \textbf{Écran de Jeu:} Affiche le personnage, les lettres, la lave, le score et les vies.
    \item \textbf{Paramètres:} Permet d'ajuster le volume, la taille de police et la taille des carrés.
    \item \textbf{Fin de Partie:} Affiche le score final et les meilleurs scores, avec une animation spéciale pour les nouveaux records.
    \item \textbf{Tutoriel:} Guide les nouveaux joueurs avec des instructions simples.
\end{itemize}

\subsection{Animations et Effets}
Le jeu intègre plusieurs animations et effets visuels:
\begin{itemize}
    \item Animations de mort (chute dans la lave ou épuisement des vies)
    \item Effets de texte animés pour les nouveaux records
    \item Affichage des rochers lorsque le joueur active un piège
\end{itemize}

\section{Aspects Techniques}

\subsection{Technologies Utilisées}
\begin{itemize}
    \item \textbf{Langage:} Python
    \item \textbf{Bibliothèque graphique:} Pygame
    \item \textbf{Multithread:} Utilisé pour les sons et les animations
    \item \textbf{Stockage:} Fichiers JSON pour les scores, XML pour les paramètres
\end{itemize}

\subsection{Architecture}
Le jeu est structuré selon une architecture orientée objet avec séparation des modèles:
\begin{itemize}
    \item \textbf{Modèles:} Player, Obstacle, Lava, Animation, etc.
    \item \textbf{Contrôleur:} Classe Game qui coordonne la logique et les interactions
    \item \textbf{Interface:} Menus et affichages gérés par des méthodes spécifiques
\end{itemize}

\section{Personnalisation}
Les joueurs peuvent personnaliser leur expérience via:
\begin{itemize}
    \item Réglages du volume pour la musique de menu, musique de jeu et effets sonores
    \item Taille de police des lettres
    \item Taille des obstacles
    \item Choix de la difficulté
    \item Nom du joueur pour les records
\end{itemize}

\end{document}